\section{Exams}\label{sec:exam}
In previous iterations of the course, exams were paper-based by necessity as university regulations made digital exams unfeasible.
Due to the COVID-19 pandemic, however, remote exams
% (either unsupervised or supervised via video conferencing software)
were allowed for the repeat exam in WS19 and both exams in WS20.
In the following, we outline how we adapted our exam process to this new reality and the advantages of our approach for students and staff.

\paragraph{Workflow for Students}

We chose the ArTEMiS~\cite{artemis} platform, previously described in \cref{sec:tech_setup_test}, for the exams.
In WS19, students were mostly unfamiliar with ArTEMiS,
but in WS20, they had been submitting their homework using the same workflow as during the exam.
Students were able to check out individual exam questions from a git repository and then work on them using their preferred programming environment.
In addition to programming exercises, the exams included theory questions and proofs, which were submitted as text using an online editor.

The exams were open-book, including online resources,
but posting questions to chats, forums, etc.\ was prohibited.
Usage of third-party code had to be cited using comments.
We think that the combination of a programming environment customised to the individual student's taste and access to online resources comes close to how programming is done in practice.
In contrast, previous exams were purely paper-based.
This limited the scope of exercises we were able to pose
since both writing and grading programs on paper is highly labour-intensive.
Moreover, we were able to expect a higher standard of correctness
since minor errors, in particular in syntax, are excusable when programming on paper but less so when students can test, or at least run, their programs before submission.

\paragraph{Grading}
Grading hundreds of exams on paper is a huge undertaking.
In previous iterations,
it took 10--20 people about 4--5 days to grade an exam.
Additionally, grading programming exercises on paper is error-prone, and it is unfeasible to digitise every paper submission and run it through a compiler.

Using ArTEMiS, we were able to use automated testing for programming exercises.
In most cases, we marked submissions passing all tests as fully correct.
Manual grading was only needed for submissions that failed some tests or did not compile.
In our view, manual grading is still necessary in those cases because it is generally not possible to write tests that are fine-grained to such a degree that every potential source of error is recognised and scored proportionally.
But also manual grading was simplified since test results provide valuable guidance when checking for errors in code.
Moreover, one could re-run tests after fixing compilation and semantic errors and deduct points accordingly.
As a result, we were able to grade an exam in 3 days.

\paragraph{Online Review}
Once graded, students have the right to review their
exams to check for errors in the grading process.
For paper-based exams at TUM, students usually had to make an
appointment for a time-slot to review their exam under
the supervision of teaching staff.
In large courses, this imposed a significant organisational overhead for both staff and students.

Using ArTEMiS, every student could simply review their exam and submit complaints through an online interface.
Following the review period,
the complaints were approved or rejected by the instructors.
This resulted in a vastly increased proportion of students reviewing their exam,
thus ensuring a higher degree of fairness in the grading process.
In addition, the quality of feedback for students improved
since
they now also got to see the test results of their submission next to the in-line grading comments made by
teaching staff.

\paragraph{Cheating}

Regulations at TUM allowed both supervised or unsupervised remote exams.
We considered supervision but ultimately decided against it. Were we to supervise the exam ourselves,
each staff member would have to keep track of 20--30 students via their webcam in a video conferencing software.
We feel that this would hardly ensure against cheating
since, for example, students would still be able to take advice from someone out of view of the camera or communicate with others online (it does not seem feasible to simultaneously supervise the camera and screen-share of 20--30 students).
There are also commercial options for exam supervision,
but it is unclear how effective they are at preventing cheating, and the options we reviewed raised significant privacy concerns.

We thus decided to rely on an honour system for our exams.
Beyond the honour pledge,
we also checked for plagiarism with Moss.
This turned up only three cases of conclusive plagiarism.
We suspect and accept that there were likely more cases of cheating that we could not catch,
so we aim to improve our anti-cheating measures in future iterations.
We also created 3--4 slight variations of each exam question
in order to make plagiarism slightly more onerous. ArTEMiS supports the creation of these variants and assigns them randomly during the exam.
